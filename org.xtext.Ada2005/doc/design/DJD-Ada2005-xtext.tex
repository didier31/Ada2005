\documentclass[a4paper]{prjdoc}
\usepackage{relsec}
\usepackage{caption}
\usepackage{listings}

\project{}
\program{Ada 2005 syntax-sensitive eclipse plugin}
\version{draft} 
\title{Design Justification Document : cross-references}
\author {Didier Garcin}
\date{February 2015, 11th}

\begin{document}

\maketitle

  \begin{asection}{Generalities}
     
     \begin{figure}\label{fig:DeclarativeItems}
     \includegraphics[scale=1]{"../../model/Declarative Items"}
     \captionof{figure}{Declarative Items} 
     \end{figure} 
     
     \begin{asection}{Accessibility rules}
     \label{sec:Accessibility rules}
     Accessibility rules deal with authorized reference between two \class{Basic Declarations} of differents \class{Library Unit Declaration} 
     and between\class{Library Unit Declarations} themselves as well. 
        \begin{enumerate}   
        \item All public\class{Declarative Items} are accessible.
        \item A private\class{Library Unit Declaration} is accessible by other private \class{Library Unit Declarations} only.
        \item To access an accessible \class{Basic Declaration} or a \class{Library Unit Declarations} itself, 
              a\class{Library Unit Declarations} must use a \class{with} clause of the declaring
              \class{Library Unit Declarations}. To limit the accessibility to private part only, \class{private with} must be used.
        \end{enumerate}
     \end{asection} %% Accessibility rules
      

     \begin{asection} {Visibility rules} 
     \label{sec:Visibility rules}             
     Visibility rules deal with declarations that can be directly referenced without the help of no special clause.
      
        \begin{enumerate} 
        \item A \class{Declarative Item} can see \class{Declarative Items}
              declared before in its \class{Library Unit Declaration}.    
        \item A \class{Declarative Item} can see the \class{Basic Declarative Items} of its parent library unit.
        \item Only \class{Declarative Items} of a \class{Package Body} can see 
              \class{Basic Declarative Items} in the private section of its \class{Package Specification}.
        \item The \class{Declarative Items} of a \class{Body} or a \class{Separate subunit} are hidden from outside.
        \end{enumerate}
     \end{asection} %% Visibility Rules
      
     \begin{asection}{Type resolution}
     \underline{Subtyping}: A subtype is a subset of its base type. By extension, it can be said that a type is its own subtype.
                     
     \underline{Derivation}: A derived type is a new type defined with the help of a type named the base type.     
   
     \underline{Universal type} : Literal constants are assigned an anonymous universal type from which all types are subtype.
   
     Type resolution's rules on primitive types:
        \begin{enumerate}
        \item Universal type x Type is Type
        \item Type x Universal type is Type
        \item Subtype of Type x Type is Type
        \item Type x Subtype of Type is Type          
        \end{enumerate}
   
     \underline{Class assignation's rule} : An object of a given class can be assigned by an object of a subclass of this class.
                                            As a result, assignement rule acts as subtyping.         
     \end{asection} %% Type resolution
   
     \begin{asection}{Subprogram's call resolution}
     A subprogram is identified by its name and by the type or class of its parameters.
     In particular, formal parameters must be assignable with the actual parameters.
     If a subprogram is a primitive\footnote{A primitive is a subprogram whom at least one parameter
     is of the type just declared before} of a type then an implicit version of this subprogram exists 
     for every derived type.      
     \end{asection} %% Subprogram's call resolution  

     \begin{asection}{Referencing different kind of object with an unique type}
        \label{sec:ReferencingUniqueType}        
        \begin{asection}{Statement of problem}            
        In Ada syntax, there are some situations when a name may designate different kind of objects :            
           \begin{itemize}
           \item function call / variable reference in expressions.
           \item subprogram's and package's libraries in with and use clauses.
           \end{itemize}
        This examples are contrary to an unique type authorized by Xtext to reference to.             
        \end{asection} %% Statement of problem
         
        \begin{asection}{General strategy}
        The rule of the unique referenced type implies to designate a representative class from which 
        it is possible to navigate to the actual referenced object.           
        \end{asection} %% General Stratery
         
        \begin{asection}{Justification of design}
         
           \begin{asection}{Solution A}
           \label{sec:SolutionA}
            
           In view of the previous section, one solution consists in use as representative, 
           a common ancestor to the different possible referenced types. 
           The actual type of the object can then be retrieved thanks to \element{instanceof} test.
            
           \end{asection} %% Solution A 
         
           \begin{asection}{Solution B}
           \label{sec:SolutionB}
            
           A second solution consists in encapsulation in a common container. 
           Assuming that only one encapsulation is valid at the same time,
           it is then possible to retrieve the actual object.
         
           Xtext demands that it can be getted the name of the object. 
           For this purpose, Xtext allows to derive one of its runtime class \class{DefaultDeclarativeQualifiedNameProvider}
           and to define \element{protected QualifiedName qualifiedName([RepresentativeType] ref)} in order to provide a name.
               
           \end{asection} %% Solution B
            
        \end{asection} %% Justification of design
         
     \end{asection} %% Referencing different kind of object with a sole type               
         
  \end{asection} %% Generalities
   
  \begin{asection}{Ada types referencement}     
      
     \begin{figure}
     \includegraphics[scale=0.81]{"../../model/Typing hierarchy"}
     \captionof{figure}{Typing hierarchy}
     \label{fig:TypingHierarchy}
     \end{figure}
   
     \begin{figure}
     \includegraphics[scale=0.90]{"../../model/References to types"}
     \captionof{figure}{References to types}
     \label{fig:ReferencesToTypes}
     \end{figure}

     \begin{asection}{Statement of problem}
      
     Figure~\ref{fig:TypingHierarchy} shows classes involved in Ada typing.
     Meanwhile and figure~\ref{fig:ReferencesToTypes} show classes that references types.
     The latters delegates to \class{Subtype Indication}
     \footnote{This attribute could be shared by these classes thanks to a common ancestor. 
     Perhaps, scoping could take advantage of.} the care to reference a type.
     \class{Subtype Indication} always refers on a type by its name; 
     hence, a cross-reference should be resolved. 
       
     \end{asection} %% Statement of problem

     \begin{asection}{The type of the reference}
     \class{Type Declaration} is not only at the top of the class hierarchy in charge of Ada typing but
     it defines \element{name} too that 
     \class{DefaultDeclarativeQualifiedNameProvider} can handle.
     As a consequence, the type of \class{Subtype Indication}::\element{subtypeMark} should be \class{Type Declaration}.            
     \end{asection} % The type of the reference 
             
     \begin{asection}{Where Ada types are declared}
     Figure~\ref{fig:TypingHierarchy} shows that type declarations 
     \class{Data Instance Declaration} are all \class{Basic Declarative Item}.
 
     So, they may appear in public or private section of a \class{Package Specification}.

     It shows as well they are \class{Declarative Item} too.
     As such, they may appear in any Ada body or \class{Block Statement}.

     As a result, all the rules of visibility and accesibility apply.
     See section~\ref{sec:Visibility rules} and \ref{sec:Accessibility rules} for details.

     \end{asection} %% Where Ada types are declared
       
     \begin{asection}{Targets of references}
     The problem exists because a same type may have more than declaration :
        \begin{itemize}
        \item \class{Full Data Type Declatation}
        \item \class{Incomplete Type Declaration}
        \item \ldots
        \end{itemize}
     The goal for \class{Subtype Indication} is to refer to the most complete definition.
     However, the most complete definition of the referenced type is the closest from \class{Subtype Indication}.
     So, reverse visiting in AST solves this purpose.
     \end{asection} %% Targets of references
  \end{asection} %% Referencing types

  \begin{asection}{Variables and functions referencement}
  By elements of program, it has to be understood :
     \begin{itemize}
     \item data
     \item functions       
     \end{itemize}
      
     \begin{figure}
     \includegraphics[scale=0.65]{"../../model/Data and Functions initially"}
     \captionof{figure}{Data and Functions : initial design}
     \label{fig:DataandFunctionsInitial}
     \end{figure}
     
     \begin{asection}{Statement of problem}
     The figure~\ref{fig:DataandFunctionsInitial} shows which kind of named elements may appear in an expressions :
        \begin{itemize}
        \item \class{Function Specification}
        \item \class{Data Instance Declaration}
        \item \class{Number Declaration}
        \item \class{Component Item}
        \item \class{Discriminant Specification}
        \end{itemize}
    
     What we learn by this list is that an expression may address various kinds of named entities and elements.
     The latters are named thanks to \class{Defining Identifier List}, \class{Function Specification} is named 
     thanks to \class{Generic Unit Declaration}. This raises some problems treated just after.       
     \end{asection} %% Statement of problem
       
     \begin{asection}{Getting the objects definition by their name}             
     The question is how to reference functions, constants and variables in \class{Expression} with an unique type of reference.
     This is not a trival question watching this in detail.
        
     Indeed, \class{Data Instance Declaration} and \class{Number Declaration} may gather several declarations. 
     The name of these instances are stored in list of \class{Defining identifier list}.
     Meanwhile, the names of functions are defined in \class{Generic unit Declaration}.
      
        \begin{asection}{Solving the problem of referencing to a list of strings as names}          
        Before all, the problem is that \class{Defining identifier list} is not a candidate for being a type of reference because 
        certainly, it defines an attribute, \element{name} of type \class{ecore::EString} but it is a list of names. 
        Hence, its type, \class{ecore::EString} as primitive type must be replaced by a class, \class{Identifier}.
        As this, it \class{Named Element} may be used as referenced type. 
        From \class{Identifier}, the definition of the element can then be accessed through containers links.          
        \end{asection} %% Solving the problem of referencing to a list of strings as names
       
        \begin{asection}{The need of unifying function's referencing with others named elements of expressions}          
        Here is the second problem to address. The common way to unify is to define a new class from which \class{Subprogram Specification}
        and \class{Identifier} derive : \class{Named Element}. Because identifiers don't include operation's signs (+, -, *, \ldots) 
        but \class{Function Specification}, yes, an intermediate class in inheritage must be added,
        \class{Direct Name} for \class{Function Specification}.                    
        \end{asection} %% The need of unifying function's referencing with others named elements of expressions
     
     The final result of these solutions is visible on figure~\ref{fig:DataandFunctions}.
     
        \begin{figure}
        \includegraphics[scale=0.59]{"../../model/Data and Functions"}
        \captionof{figure}{Data and Functions : final design}
        \label{fig:DataandFunctions}
        \end{figure}          
     
     \end{asection} %% Getting the objects definition by their name

  \end{asection} %% Variables and functions referencement
    
  \begin{asection}{imports : ``with'' and ``use'' clauses}
    
     \begin{figure}
     \includegraphics[scale=0.85]{"../../model/Library units"}
     \captionof{figure}{Library units}
     \label{fig:Library units}
     \end{figure} 
    
     \begin{asection}{Statement of problem}    
     Figure \ref {fig:Library units} gives an overview of the problem.
    
     On one part, \class{With clause} references a \class{Library unit} by its name.
     On the other part, \class{Library unit} doesn't have a \element{name} attribute.    
     Instead, \element{name} can be retrieved in its descendants or in one of their parts.    
     \end{asection} %% Statement of problem
    
     \begin{asection}{Justification of design}
     Provided that \element{name} attribute is dispatched in class hierarchy of \class{Library unit},
     the solution described in section \ref{sec:SolutionB} is the most suitable.    
     \end{asection} %% Justification of design
	 
  \end{asection} %% imports : With and Use clauses
  
  \begin{asection}{Class \class{Name}}
     \begin{asection}{Statement of problem}
     The syntax by the Ada reference manual tells a \class{Name} is :
        \begin{itemize}
        \item either a direct name       (identifier or operation symbol)
        \item or an explicit dereference (deferencement of an access)
        \item or an indexed component    (a cell of an anrray)
        \item or a slice                 (a subarray)
        \item or a selected component    (the component of a record)
        \item or an attribute reference  (of a name)
        \item or a type conversion      
        \item or a function call
        \item or a character literal     (to designate ASCII indexes by their character literals. I can see nothing else.)
        \end{itemize}
  
     \ref{fig:Class Name used for programs} shows that Name is used to designate entries (with class \class{Requeue Statement}), 
     tasks and exceptions too.
  
        \begin{figure}
        \includegraphics[scale=1.0]{"../../model/Class Name used for data"}
        \captionof{figure}{Class Name used for data}
        \label{fig:Class Name used for data}
        \end{figure}
  
        \begin{figure}
        \includegraphics[scale=0.99]{"../../model/Class Name used for programs"}
        \captionof{figure}{Class Name used for programs}
        \label{fig:Class Name used for programs}
        \end{figure} 
     
        \begin{figure}
        \includegraphics[scale=0.91]{"../../model/Class Name used for types"}
        \captionof{figure}{Class Name used for types}
        \label{fig:Class Name used for types}
        \end{figure} 
  
     As a result, a Name designates either a data, a type, a subprogram, an entry, a task or an exception.
     
     This point is a problem for references because identifiers in class \class{Name} must refer to an unique type.                  
     \end{asection} %% Statement of problem 
  
     \begin{asection}{General strategy}
     
        \begin{asection}{Solution A}
        A solution consists in applying strategy of section \ref{sec:SolutionA}. 
        This assumes to define a common ancestor to these classes letting 
        the cross-reference find by name the correct object.       
        \end{asection}
        
        \begin{asection}{Solution B}
        \class{Name} is very generic and used to designate many classes and all its syntax is not always used.
        Hence, another solution consists in splitting \class{Name} in several \class{Name} specialized classes.
        
           \begin{asection}{Method for \class{Name} specialization}
           For each class having an attribute of type Name, 
           determine the type of the referenced object.
           
           For this type, localize the attribute \element{name}.
           
           If the attribute \element{name} is defined in an ancestor,
           inventory its descendants.
           
           For each of its descendant, determine the general form of its qualified name based on the syntax and semantic of its Ada declaration(s)
           and if a language-defined attribute has its type.
           
           Create a new class \class{Name} rid of unnecessary syntactic constructions.
                       
           \end{asection} %% Method for \class{Name} specialization
           
           The following sections apply this algorithm.
           
           \begin{asection}{Name designating an exception's name}
           \ref{fig:Class Name used for programs} shows \class{Raise Statement} and \class{Exception Declaration}
           that designate exception by its name of type \class{Name}.
           
           The declaration of a exception is immuable : an identifier followed by the keyword \textit{exception}.
           This means that its qualified name is always the including packages names followed by its identifier :
           Exception qualified name : (Package identifier)* Exception's identifier.           
           \end{asection} %% Name designating an exception's name

           \begin{asection}{Name designating a task}
           Tasks can be declared as data :
              \begin{enumerate}
              \item They can be indexed
              \item referenced by an access variable
              \item discriminated 
              \end{enumerate}              
              
           It doesn't exist language-defined attribute of type task. 
           Its name is an identifier.
              
           As a result, \class{Task Name} doesn't need :
              \begin{itemize}
              \item Attribute designator : no predefined attribute is a task.
              \item operator symbol neither character literal literals 
              \end{itemize}              
           \end{asection} %% Name designating a task
           
           \begin{asection}{Name designating a procedure or an entry call}
           
           As a procedure and entry calls are syntactically indistiguable, their call-name are the same.
           Here, optimizations are analyzed separately and unified at the end of this section. 
           
              \begin{asection}{Name designating an entry}
           
              An entry can :
                 \begin{enumerate}
                 \item be indexed 
                 \item have parameters (i.e: a formal part)
                 \end{enumerate}              
              
                 It doesn't exist language-defined attribute of type entry.
                 Its name is an identifier.
                 It can't be referenced by an Ada access.
              
             As a result, \class{Entry Name} doesn't need :
                 \begin{itemize}
                 \item Attribute designator : no predefined attribute is a task.
                 \item operator symbol neither character literal literals
                 \item .all as an entry can't be designated by a type. 
                 \end{itemize}              
           
              \end{asection} %%Name designating an entry
           
              \begin{asection}{Name designating a procedure}
         
                 A procedure can :
                    \begin{enumerate}
                    \item be referenced by an Ada access
                    \item be indexed indirectly through an Ada access.
                    \item have parameters (i.e: a formal part)
                    \item can adopt the language-defined attribute's notation
                    \end{enumerate}              
             
                Its name is an identifier.
              
                As a result, \class{Procedure Name} doesn't need :
                   \begin{itemize}
                   \item operator symbol neither character literal literals 
                   \end{itemize}  
         
              \end{asection} %% Name designating a procedure
              
              \begin{asection}{Conclusion}
              
              As a result, A procedure/entry call-name doesn't need :
                   \begin{itemize}
                   \item operator symbol neither character literal literals
                   \end{itemize}
                   
              \end{asection}          
              
           \end{asection} %% Name designating a procedure or an entry call

        \end{asection} %% Solution B

     \end{asection} %% General Strategy
  
  \end{asection} %% Class Name
  
  \begin{asection}{Reference to Ada entry}
     
     \begin{figure}
     \includegraphics[scale=0.78]{"../../model/Entry declaration"}
     \captionof{figure}{Entry declaration}
     \label{fig:Entry declaration}
     \end{figure}  
        
     \begin{figure}
     \includegraphics[scale=0.78]{"../../model/Entry call"}
     \captionof{figure}{Entry call}
     \label{fig:Entry call}
     \end{figure}
     
     Figure \ref{fig:Entry call} shows the classes that reference entries :
        \begin{enumerate}
        \item \class{Accept Statement}
        \item \class{Procedure Or Entry Call Statement}
        \item \class{Entry Body}
        \end{enumerate} 
     
     \begin{asection}{The scope}
     
     \end{asection} %% The scope
  
  \end{asection} %% Reference to Ada entry
  
  \begin{asection}{Linking with Xtext}
     \begin{asection}{Names and namespaces}
     Linking deals with referencing by name.
     To ensure names are unique, container's names are used as namespaces in order to disambiguate.
     For instance, the local variable of a method is preceded by the name of this method which 
     is preceded by the name of the class and so on. 
     
     Concatenating these names is performed by descendants of \class{IQualifiedNameProvider}. 
     By default, \class{DefaultDeclarativeQualifiedNameProvider} is the implementator.
     This function of this class is based on attributes \element{name} of the ecore classes of the meta-model.  
     Nevertheless, the xtext grammar of the user could not respect for good reasons the convention used by the default implementation.
     That's why it is possible to derive \class{DefaultDeclarativeQualifiedNameProvider} to adapt the function. 
     More, do not forget to add in the descendant of \class{DefaultRuntimeModule} :
     \begin{lstlisting}
public Class<? extends org.eclipse.xtext.naming.IQualifiedNameProvider> 
bindIQualifiedNameProvider() 
{
   return YourQualifiedNameProvider.class;
}
     \end{lstlisting}
     A name preceded by its containers names is called a \class{QualifiedName}.    
       
     \end{asection} %% Names
     
     \begin{asection}{Named objects container}
     All named objects are stored in a global container called the index.
     
     %% TODO : Complete the chapter
     
     \end{asection} %% Ada objects container
  
  \end{asection} %% Cross-referencing by Xtext 
\end{document}